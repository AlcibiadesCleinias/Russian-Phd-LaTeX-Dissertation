
{\annotationtext} В данной работе основной целью является построение модели двойной пористой среды, в которой учитывается зависимость интенсивности массообмена между подсистемами матрицы и трещины от параметра разрушаемости, который характеризует и моделирует процесс растрескивания матрицы под действием разницы гидростатического давления в трещинах и порового давления в блоках матрицы (если поровое давление в матрицах больше гидростатического, говорят об аномально высоком пластовом давлении). Также основной задачей является изучение влияния аномально высокого пластового давления в полученной физической модели на дебит со скважины.

Полученная модель двойной пористой среды с упругим хрупким скелетом моделирует аномально высокое пластовое давление, которое может являться источником образования вторичных трещин в системе матриц, что приводит к увеличению интенсивности массообмена подсистем и, как результат, сказывается на существенном увеличении дебита со скважины в сравнении с традиционными моделями двойной пористой среды. Численный расчет демонстрирует решение задачи о производительности длинной цилиндрической скважины.

Основными рекомендациями к данной работе являются следующие: построение лабораторного эксперимента и сравнение с численными исследованиями полученной модели.
