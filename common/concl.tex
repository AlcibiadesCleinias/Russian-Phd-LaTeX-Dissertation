%% Согласно ГОСТ Р 7.0.11-2011:
%% 5.3.3 В заключении диссертации излагают итоги выполненного исследования, рекомендации, перспективы дальнейшей разработки темы.
%% 9.2.3 В заключении автореферата диссертации излагают итоги данного исследования, рекомендации и перспективы дальнейшей разработки темы.
\begin{enumerate}
  \item Разработана термодинамически согласованная модель двойной пористой среды с трещиноватым скелетом, отличительной особенностью которой является учет влияния аномально высокого пластового давления  на интенсивность массообмена между континуумами матриц и трещин. Для этого в модель введена внутренняя переменная (параметр разрушения), характеризующая процесс разрушения низкопроницаемой матрицы вследствие комбинации процессов, связанных с изменением как порового давления, так и с изменением напряженного состояния скелета.
  \item Проведен качественный и количественный анализ параметров модели.
  \item Численные исследования показали, что аномально высокое пластовое давление может приводить к разрушению в низкопроницаемой матрице вблизи окрестности скважины. Подобные условия возникают вследствие возмущений, связанных с резким уменьшением давления в трещинах по сравнению с поровым давлением в матрицах, а также связанных с изменением полного напряжения в скелете в окрестности скважины. В результате этих возмущений фактически увеличивается эффективный радиус скважины (увеличивается размер области дренирования).
  \item Продемонстрирована принципиальная возможность разработки низкопроницаемых трещиновато-пористых месторождений,~--- используя только энергию аномально высокого пластового давления.

\end{enumerate}
